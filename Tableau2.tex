% \documentclass[a4paper,12pt,twoside]{report}
% 
% % Pour inclure des pages issus d'un autre pdf
% \usepackage{pdfpages}
% 
% 
% \usepackage[utf8x]{inputenc}
% %\usepackage[latin1]{inputenc}
% \usepackage[T1]{fontenc}
% \usepackage[francais]{babel}
% 
% %\usepackage{layout}
% \usepackage{geometry}
% \geometry{%
% a4paper,
% body={180mm,265mm},
% left=20mm,top=15mm,
% headheight=7mm,headsep=4mm,
% marginparsep=4mm,
% marginparwidth=11mm}
% 
% \usepackage{setspace} % Interligne
% %\singlespacing
% %\onehalfspacing
% \doublespacing
% 
% \setlength{\parindent}{1cm} %Pour le retrait du paragraphe
% 
% \usepackage{soul} % Pour souligner ou barrer du texte
% \usepackage{ulem}
% 
% \usepackage{textcomp}
% 
% %\usepackage{eurosym}
% 
% %\usepackage{bookman} % Différents packs de police
% %\usepackage{charter}
% %\usepackage{newcent}
% %\usepackage{lmodern}
% %\usepackage{mathpazo}
% %\usepackage{mathptmx}
% 
% %\usepackage{url}
% 
% %\usepackage{verbatim}
% 
% %\usepackage{moreverb}
% 
% %\usepackage{listings}
% 
% \usepackage{fancyhdr} % en tête et pied de pages
% 
% \usepackage{wrapfig} % pour encadrer les figures avec du texte
% \usepackage{graphicx}
% 
% \usepackage{multicol} % Pour utiliser l'environnement multicol
% \setlength\columnseprule{.4pt} % Pour mettre un trait séparateur de colonnes\usepackage{multirow} % Pour fusionner les lignes dans les tableaux
% 
% 
% 
% \usepackage{array} % Pour centrer les lignes en hauteur dans un tableau
% \usepackage{rotating} % Pour tourner un tableau
% 
% \usepackage[table]{xcolor} %Pour colorer des cellules
% 
% \usepackage{xcolor}
% \usepackage{colortbl}
% 
% \usepackage{amsmath}
% \usepackage{amssymb}
% \usepackage{mathrsfs}
% 
% %\usepackage{asmthm}
% %\usepackage{makeidx}
% 
% \ifpdf %Pour mettre des hyperliens dans les pdfs
% \usepackage[pdftex=true,
% hyperindex=true,
% colorlinks=true]{hyperref}
% \else
% \usepackage[hypertex=true,
% hyperindex=true,
% colorlinks=false]{hyperref}
% \fi

%\begin{document}
% \begin{sidewaystable}
\setlength{\tabcolsep}{1pt}
\centering
\caption{\textbf{Résultats des comptages dans les différentes Raphés bulbaires}}
\setlength\minrowclearance{6pt}

{%
\newcommand{\mc}[3]{\multicolumn{#1}{#2}{#3}}

\newcolumntype{A}{%
>{\columncolor[gray]{.9}}%
c|}

\newcolumntype{B}{%
>{\columncolor[gray]{1}}%
c|}

{\protect\parbox[t]{25cm}{\singlespacing
\begin{small}
\textbf{(A)} Comptage dans la partie rostrale des noyaux du Raphé Magnus (RMg) et Latéral Paragigantocellulaire (LPGi).
\end{small}}}
\begin{scriptsize}
\begin{center}
\begin{tabular}[c]{||BBc|c|c|c|c|c|c|c|c|c|c|c|c|c|c|c||}
\hline\hline
Rostral		&  		& \mc{4}{A}{\textbf{LPGi Gauche}} 										& \mc{4}{A}{\textbf{RMg Gauche}} 								& \mc{4}{A}{\textbf{RMG droit}}									 & \mc{4}{A}{\textbf{LPGi Droit}}										 \\\hline
 		& \textbf{n} 	& \textbf{5-HT} & \textbf{Fos} 			& \textbf{5-HT+Fos} 		& \textbf{\%} 			& \textbf{5-HT} & \textbf{Fos} 	& \textbf{5-HT+Fos} 		& \textbf{\%} 			& \textbf{5-HT} & \textbf{Fos} 	& \textbf{5-HT+Fos} 		 & \textbf{\%} 			 & \textbf{5-HT} & \textbf{Fos} 		  & \textbf{5-HT+Fos} 		  & \textbf{\%} 		 \\\hline
\textbf{33~°C} 	& 9 		& 26 (1,1) 	& 1,9 (0,3)			& 6,3 (1,2) 			& 24 (4,4)			& 32 (1) 	& 4,9 (0,5) 	& 3 (0,5) 			& 9,2 (1,4) 			& 31 (1,4) 	& 4,4 (0,7) 	& 2,6 (0,4) 			 & 8,5 (1,4) 			 & 25 (0,8) 	 & 2,3 (0,4) 			  & 5,9 (0,9) 			  & 23 (3,5)			 \\\hline
\textbf{46~°C} 	& 5 		& 25 (0,6) 	& 2,7 (0,4) 			& 6,9 (1,2) 			& 27 (4,5)			& 30 (1,6) 	& 7,5 (1,4) 	& 2,5 (0,5) 			& 12 (2,1) 			& 29 (1,4) 	& 6,7 (1,4) 	& 2 (0,6) 			 & 9,9 (2,9)			 & 25 (1,2) 	 & 3,1 (0,6) 			  & 5,5 (1,2) 			  & 22 (4,1)			 \\\hline
\textbf{48~°C} 	& 5 		& 28 (0,8) 	& \underline{\textbf{4,2 (0,9)}}& \underline{\textbf{12 (1,4)}}	& \underline{\textbf{41 (4,3)}}	& 32 (1,7) 	& 7,2 (1,3) 	& 3,7 (0,5)			& 11 (1,6) 			& 31 (1,1) 	& 6 (0,9) 	& 3,3 (0,8) 			 & 10 (2,2) 			 & 26 (1,5) 	 & \underline{\textbf{3,9 (0,4)}} & \underline{\textbf{9,6 (1,4)}}& \underline{\textbf{39 (7,4)}}\\\hline
\textbf{52~°C} 	& 10 		& 27 (0,7) 	& \underline{\textbf{4 (0,6)}} 	& \underline{\textbf{16 (1,5)}} & \underline{\textbf{57 (4,8)}}	& 31 (1,2) 	& 7,2 (1) 	& \underline{\textbf{6,7 (0,4)}}& \underline{\textbf{22 (1,3)}} & 31 (0,8) 	& 5,8 (1) 	& \underline{\textbf{6,3 (0,6)}} & \underline{\textbf{20 (1,7)}} & 25 (1)   	 & \underline{\textbf{4 (0,7)}}   & \underline{\textbf{13 (1,4)}} & \underline{\textbf{49 (3,8)}}\\\hline\hline
\end{tabular}
\end{center}
\end{scriptsize}

{\protect\parbox[t]{25cm}{\singlespacing
\begin{small}
\textbf{(B)} Comptage dans la partie caudale des noyaux du Raphé Magnus (RMg) et Latéral Paragigantocellulaire (LPGi).
\end{small}}}
\begin{scriptsize}
\begin{center}
\begin{tabular}[c]{||BBc|c|c|c|c|c|c|c|c|c|c|c|c|c|c|c||}\hline\hline
Rostral		&  		& \mc{4}{A}{\textbf{LPGi Gauche}} 					& \mc{4}{A}{\textbf{RMg Gauche}} 						& \mc{4}{A}{\textbf{RMG droit}}					& \mc{4}{A}{\textbf{LPGi Droit}}\\\hline
 		& \textbf{n} 	& \textbf{5-HT} & \textbf{Fos} 	& \textbf{5-HT+Fos} 	& \textbf{\%} 	& \textbf{5-HT} & \textbf{Fos} 	& \textbf{5-HT+Fos} 	& \textbf{\%} 		& \textbf{5-HT} & \textbf{Fos} 	& \textbf{5-HT+Fos} 	& \textbf{\%} 	& \textbf{5-HT} & \textbf{Fos}    & \textbf{5-HT+Fos}	& \textbf{\%}	\\\hline
\textbf{33~°C} 	& 9 		& 25 (1,6) 	& 3,6 (0,4)	& 1,9 (0,4)		& 7,6 (1,7)	& 37 (2) 	& 3,8 (0,5) 	& 1,1 (0,8) 		& 2,8 (0,7) 		& 38 (2,8) 	& 4,1 (0,7) 	& 0,8 (0,2) 		& 2,4 (0,8) 	& 28 (1,4) 	& 3,1 (0,5) 	  & 1,9 (0,3)		& 6,9 (1,1)	\\\hline
\textbf{46~°C} 	& 5 		& 30 (1,3) 	& 3,3 (0,6) 	& 1,9 (0,4)		& 6 (1,3)	& 38 (1,8) 	& 3,1 (0,3) 	& 0,9 (0,2) 		& 2,5 (0,6) 		& 39 (3,4) 	& 4,3 (0,7) 	& 1,4 (0,3) 		& 3,6 (0,8)	& 27 (2,1) 	& 4 (0,5) 	  & 2,4 (0,6) 		& 9,4 (2,6)	\\\hline
\textbf{48~°C} 	& 5 		& 30 (1,7) 	& 3,3 (0,7)	& 1,9 (0,3) 		& 6,5 (1) 	& 37 (1,1) 	& 2,9 (0,4) 	& 0,8 (0,1)		& 2,2 (0,3) 		& 40 (2,7) 	& 3,7 (0,4) 	& 0,8 (0,2) 		& 2,1 (0,5) 	& 27 (2,7) 	&  2,7 (0,6)	  &  2,1 (0,3)		&  7,9 (0,6) 	\\\hline
\textbf{52~°C} 	& 10 		& 28 (2) 	& 3,2 (0,4) 	& 2,3 (0,4) 		& 8,7 (1,8) 	& 38 (1,3) 	& 3,2 (0,5) 	&  1,2 (0,2) 		&  3 (0,5)  		& 37 (2,2) 	& 3,5 (0,4) 	&  1,3 (0,2)  		&  3,6 (0,6)  	& 26 (1,7)   	&  2,5 (0,3)   	  &  2,2 (0,4) 	  	&  9,2 (2)	\\\hline\hline
\end{tabular}
\end{center}
\end{scriptsize}

{\protect\parbox[t]{25cm}{\singlespacing
\begin{small}
\textbf{(C)} Comptage dans les Raphés Dorsalis (RDr), Obscurus (Rob), Pallidus (RPa) rostral et caudal (Tableau C).
\end{small}}}
\begin{scriptsize}
\begin{center}
\begin{tabular}[c]{||BBc|c|c|c|c|c|c|c|c|c|c|c|c|c|c|c||}\hline\hline
Rostral		&  		& \mc{4}{A}{\textbf{RDr}} 						& \mc{4}{A}{\textbf{RPa (Rostral)}} 						& \mc{4}{A}{\textbf{RPa (Caudal)}}					& \mc{4}{A}{\textbf{LPGi Droit}}\\\hline
 		& \textbf{n} 	& \textbf{5-HT} & \textbf{Fos} 	& \textbf{5-HT+Fos} 	& \textbf{\%} 	& \textbf{5-HT} & \textbf{Fos} 	& \textbf{5-HT+Fos} 	& \textbf{\%} 		& \textbf{5-HT} & \textbf{Fos} 	& \textbf{5-HT+Fos} 	& \textbf{\%} 	& \textbf{5-HT} & \textbf{Fos}    & \textbf{5-HT+Fos}				& \textbf{\%}	\\\hline
\textbf{33~°C} 	& 9 		& 262 (19) 	& 26 (3,2)	& 1,9 (0,4)		& 0,4 (0,1)	& 15 (1,2) 	& 20 (2,2) 	& 1,1 (0,2) 		& 8,2 (1,7) 		& 13 (1,2) 	& 1,9 (0,3) 	& 1,4 (0,3) 		& 14 (1,8) 	& 47 (1,9) 	& 8,8 (0,9) 	  & 2,7 (0,4)					& 5,8 (1)	\\\hline
\textbf{46~°C} 	& 5 		& 253 (1,3) 	& 29 (5,2) 	& 3 (1,2)		& 0,7 (0,3)	& 17 (2,8) 	& 24 (2,1) 	& 1,6 (0,3) 		& 9,4 (1,6) 		& 14 (0,9) 	& 2,2 (0,3) 	& 2 (0,1) 		& 14 (0,9)	& 45 (1,4) 	& 7,7 (0,9) 	  & 2,7 (0,5) 					& 6 (1,1)	\\\hline
\textbf{48~°C} 	& 5 		& 233 (1,8) 	& 26 (3,2)	& 3,1 (0,7) 		& 0,6 (0,1) 	& 14 (2,9) 	& 25 (2,2) 	& 1,4 (0,3)		& 12 (3,6) 		& 14 (1,3) 	& 1,8 (0,5) 	& 1,3 (0,2) 		& 9,8 (1,6) 	& 47 (1,4) 	&  8,0 (0,7)	  &  3,2 (0,4)					& 6,7 (0,9) 	\\\hline
\textbf{52~°C} 	& 10 		& 251 (17) 	& 27 (4,8) 	& 3,1 (0,6) 		& 0,6 (0,1) 	& 13 (1,5) 	& 23 (2) 	&  1,3 (0,2) 		&  11 (1,5)  		& 13 (1,6) 	& 2,7 (0,4) 	&  1,8 (0,3)  		&  12 (0,6)  	& 49 (2,1)   	&  9,9 (0,9)   	  &  \underline{\textbf{4,4 (0,3)}} 	  	&  \underline{\textbf{8 (0,7)}}	\\\hline\hline
\end{tabular}
\end{center}
\end{scriptsize}


{\protect\parbox[t]{25cm}{
\begin{scriptsize}
\singlespacing
\noindent Les résultats pour ces trois tableaux sont exprimés en nombre moyen de neurones par coupe et SEM (entre parenthèses). Les nombres en gras soulignés correspondent aux résultats statistiquement significatifs : $p < 0,05$ (Test de Student). 5HT~: neurone sérotoninergique~; Fos~: neurone positivement marqué par la protéine c-Fos~; 5HT+Fos et \%~: nombre absolu et pourcentage de neurones sérotoninergique doublement marqués
\end{scriptsize}}}
}

% \end{sidewaystable}
%\end{document}
