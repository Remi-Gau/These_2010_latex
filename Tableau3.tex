% \documentclass[a4paper,12pt,twoside]{report}
% 
% \usepackage[utf8x]{inputenc}
% %\usepackage[latin1]{inputenc}
% \usepackage[T1]{fontenc}
% \usepackage[francais]{babel}
% 
% \usepackage{setspace} % Interligne
% %\singlespacing
% %\onehalfspacing
% \doublespacing
% 
% \usepackage{soul} % Pour souligner ou barrer du texte
% \usepackage{ulem}
% 
% \usepackage{textcomp}
% 
% \usepackage{wrapfig} % pour encadrer les figures avec du texte
% \usepackage{graphicx}
% 
% \usepackage{multicol} % Pour utiliser l'environnement multicol
% \setlength\columnseprule{.4pt} % Pour mettre un trait séparateur de colonnes\usepackage{multirow} % Pour fusionner les lignes dans les tableaux
% 
% \usepackage{array} % Pour centrer les lignes en hauteur dans un tableau
% \usepackage{rotating} % Pour tourner un tableau
% 
% \usepackage[table]{xcolor} %Pour colorer des cellules
% 
% \usepackage{xcolor}
% \usepackage{colortbl}
% 
% 
% \begin{document}
% \begin{table}
\centering
{%
\newcommand{\mc}[3]{\multicolumn{#1}{#2}{#3}}

\newcolumntype{A}{%
>{\centering\bfseries}%
m{5cm}}

\newcolumntype{B}{%
>{\centering}%
m{1cm}}

\begin{center}
\caption{\textbf{Décharges basales des neurones sérotoninergiques du groupe B3 selon leurs réponses à une stimulation nociceptive thermique}}

\bigskip 

\begin{tabular}{|A|B|m{4.5cm}|m{4.5cm}|}
\hline
Réponse à une stimulation nociceptive	& \textbf{n}	& \begin{center}\textbf{Fréquence (Hz)}\end{center}	& \begin{center}\textbf{CV} \end{center} \\ \hline
Activés 				& 27 		& 1.55 $\pm$ 0.12 (0.37 - 2.93) 				& 0.45 $\pm$ 0.03 (0.14 - 0.73)		 \\ \hline
Inhibés					& 7 		& 1.06 $\pm$ 0.10 (0.64 - 1.35) 				& 0.26 $\pm$ 0.02 (0.20 - 0.33)		 \\ \hline
Pas de réponse 				& 3 		& 0.65 $\pm$ 0.10 (0.48 - 0.82) 				& 0.35 $\pm$ 0.03 (0.31 - 0.41)		 \\ \hline
Total 					& 37 		& 1.38 $\pm$ 0.10 						& 0.41 $\pm$ 0.02				 \\ \hline
\end{tabular}
\end{center}

{\protect\parbox[t]{18cm}{
\small Les valeurs sont exprimées sous forme de moyenne $$\pm$$ SEM. Entre parenthèses sont indiquées les valeurs extrêmes observées pour chaque groupe.
}}

}
% \end{table}
% \end{document}
