% \documentclass[a4paper,12pt,twoside]{report}
% 
% \usepackage[utf8x]{inputenc}
% %\usepackage[latin1]{inputenc}
% \usepackage[T1]{fontenc}
% \usepackage[francais]{babel}
% 
% \usepackage{setspace} % Interligne
% %\singlespacing
% %\onehalfspacing
% \doublespacing
% 
% \usepackage{soul} % Pour souligner ou barrer du texte
% \usepackage{ulem}
% 
% \usepackage{textcomp}
% 
% \usepackage{wrapfig} % pour encadrer les figures avec du texte
% \usepackage{graphicx}
% 
% \usepackage{multicol} % Pour utiliser l'environnement multicol
% \setlength\columnseprule{.4pt} % Pour mettre un trait séparateur de colonnes\usepackage{multirow} % Pour fusionner les lignes dans les tableaux
% 
% \usepackage{array} % Pour centrer les lignes en hauteur dans un tableau
% \usepackage{rotating} % Pour tourner un tableau
% 
% \usepackage[table]{xcolor} %Pour colorer des cellules
% 
% \usepackage{xcolor}
% \usepackage{colortbl}
% 
% 
% \begin{document}
% \begin{sidewaystable}

\setlength{\tabcolsep}{1pt}
\centering
\caption{\textbf{Réponses des neurones 5-HT du RMg à la morphine administrée par voie systémique}}
\setlength\minrowclearance{6pt}

\newcolumntype{A}{%
>{\centering}%
m{2.8cm}}

\newcolumntype{D}{%
>{\centering\bfseries}%
m{2.8cm}}

{%
\newcommand{\mc}[3]{\multicolumn{#1}{#2}{#3}}

\begin{tabular}{|A|A|m{6.5cm}|A|D|D|D|c}
\hline
\textbf{Études}			&\textbf{Modèle expérimental}				&\centering\textbf{Identification des neurones}										&\textbf{Morphine}		& Insensibles		&Inhibés	&Activés&	\\ \hline
Auerbach et al. \newline1985 	&Chat éveillé 						& Pattern de décharge \newline\footnotesize (lent, régulier, inactifs pendant le sommeil) \newline Pharmacologique 	&2 mg/kg \newline i.p. 		& \multicolumn{2}{c|}{\textbf{15}} 	&1	&	\\ \hline
Chiang \& Pan\newline1985	&Rat anesthésié \footnotesize uréthane 1,1 g/kg 	& Pattern de décharge \newline\footnotesize (lent, régulier) \newline Conduction lente 					&5 mg/kg \newline i.p. 		& 6 		& 3 			&1	&	\\ \hline
Gao et al. \newline1998 	&Rat anesthésié \footnotesize halothane 1\% 		& Pattern de décharge \newline\footnotesize (lent, régulier) \newline Immunohistochimique 				&0,5-10 mg/kg \newline s.c. 	& 20		& 6 			&6	&	\\ \hline
\end{tabular}

}
% \end{sidewaystable}
% \end{document}
